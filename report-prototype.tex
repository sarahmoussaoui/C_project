\documentclass[a4paper, 12pt]{article}
\usepackage[utf8]{inputenc}
\usepackage{amsmath}
\usepackage{graphicx}
\usepackage{booktabs}

\title{Rapport sur l'Étude et la Comparaison des Structures de Données et l'Optimisation du Tri par Tas}
\author{Nom de l'auteur}
\date{\today}

\begin{document}
\maketitle

\tableofcontents % Table of contents
\newpage % Start a new page
\section{Introduction}
Ce rapport présente une étude approfondie des structures de données, en particulier les listes doublement chaînées, les arbres binaires de recherche (BST), les B-trees et les tas (heap). L'objectif est de comparer les opérations élémentaires ainsi que d'optimiser l'algorithme de tri par tas.

\section{Structures de Données}
\subsection{Listes Doublement Chaînées}
\subsubsection{Description de la Structure}
Description détaillée des listes doublement chaînées.

\subsubsection{Explication des Algorithmes}
\paragraph{Insertion}
Pseudocode pour l'algorithme d'insertion.

\paragraph{Suppression}
Pseudocode pour l'algorithme de suppression.

\paragraph{Recherche}
Pseudocode pour l'algorithme de recherche.

\subsubsection{Calcul de la Complexité Théorique}
Analyse de la complexité temporelle et spatiale.

\subsubsection{Expérimentation}
\paragraph{Résultats}
Résultats obtenus de l'expérimentation.

\paragraph{Comparaison Complexité Théorique et Expérimentale}
Comparaison des complexités théorique et expérimentale.

\paragraph{Présentation des Résultats}
Tableaux et graphiques des résultats.

\paragraph{Analyse Critique}
Analyse critique des résultats obtenus.

\subsection{Arbres Binaires de Recherche (BST)}
\subsubsection{Description de la Structure}
Description détaillée des BST.

\subsubsection{Explication des Algorithmes}
\paragraph{Insertion}
Pseudocode pour l'algorithme d'insertion.

\paragraph{Suppression}
Pseudocode pour l'algorithme de suppression.

\paragraph{Recherche}
Pseudocode pour l'algorithme de recherche.

\subsubsection{Calcul de la Complexité Théorique}
Analyse de la complexité temporelle et spatiale.

\subsubsection{Expérimentation}
\paragraph{Résultats}
Résultats obtenus de l'expérimentation.

\paragraph{Comparaison Complexité Théorique et Expérimentale}
Comparaison des complexités théorique et expérimentale.

\paragraph{Présentation des Résultats}
Tableaux et graphiques des résultats.

\paragraph{Analyse Critique}
Analyse critique des résultats obtenus.

\subsection{B-trees}
\subsubsection{Description de la Structure}
Description détaillée des B-trees.

\subsubsection{Explication des Algorithmes}
\paragraph{Insertion}
Pseudocode pour l'algorithme d'insertion.

\paragraph{Suppression}
Pseudocode pour l'algorithme de suppression.

\paragraph{Recherche}
Pseudocode pour l'algorithme de recherche.

\subsubsection{Calcul de la Complexité Théorique}
Analyse de la complexité temporelle et spatiale.

\subsubsection{Expérimentation}
\paragraph{Résultats}
Résultats obtenus de l'expérimentation.

\paragraph{Comparaison Complexité Théorique et Expérimentale}
Comparaison des complexités théorique et expérimentale.

\paragraph{Présentation des Résultats}
Tableaux et graphiques des résultats.

\paragraph{Analyse Critique}
Analyse critique des résultats obtenus.

\subsection{Tas (Heap)}
\subsubsection{Description de la Structure}
Description détaillée des tas.

\subsubsection{Explication des Algorithmes}
\paragraph{Insertion}
Pseudocode pour l'algorithme d'insertion.

\paragraph{Suppression}
Pseudocode pour l'algorithme de suppression.

\paragraph{Recherche}
Pseudocode pour l'algorithme de recherche.

\subsubsection{Calcul de la Complexité Théorique}
Analyse de la complexité temporelle et spatiale.

\subsubsection{Expérimentation}
\paragraph{Résultats}
Résultats obtenus de l'expérimentation.

\paragraph{Comparaison Complexité Théorique et Expérimentale}
Comparaison des complexités théorique et expérimentale.

\paragraph{Présentation des Résultats}
Tableaux et graphiques des résultats.

\paragraph{Analyse Critique}
Analyse critique des résultats obtenus.

\section{Tri par Tas}
\subsection{Construction du Tas}
Détails sur les deux algorithmes de construction du tas.

\subsection{Tri du Tas}
Explication de l'algorithme de tri par tas.

\subsection{Comparaison des Algorithmes de Tri par Différentes Constructions}
Comparaison des performances des algorithmes de tri en fonction des constructions du tas.

\section{Conclusion Générale}
Résumé des résultats et recommandations.

\section{Références}
Incluez toutes les références utilisées pour la rédaction du rapport.

\section{Contributions des Membres du Groupe}
Indiquez les contributions de chaque membre du groupe à la réalisation du rapport.

\end{document}
